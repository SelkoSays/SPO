\documentclass[a4paper,notitlepage]{article}
\usepackage[utf8]{inputenc}
\usepackage{minted}
\usepackage{cite}
\usepackage{svg}
\usepackage{hyperref}
\usepackage{url}
\usepackage{hyphenat}
\usepackage[slovene]{babel}
\usepackage{titling}
\usepackage{tikz}
\usepackage{fvextra}
\usepackage{dirtree}

\DefineVerbatimEnvironment{mateusz}
  {Verbatim}
  {fontfamily=\rmdefault,breaklines,breaksymbolleft={}}
% tlmgr install transparent titling svg upquote minted latexindent hyphenat dirtree
\tikzset{%
    baseline,
    inner sep=2pt,
    minimum height=12pt,
    rounded corners=2pt
}
\newcommand{\inline}[1]{\mbox{% added this percent
        \ttfamily
        \tikz \node[anchor=base,fill=black!12]{#1};% added this percent
    }}
\newcommand{\code}[1]{\mintinline[breaklines]{c}{#1}}

\hbadness=500000
\tolerance=10000

\usemintedstyle{sas}


\author{Jan Mrak \\
    \small Fakulteta za računalništvo in informatiko, Univerza v Ljubljani}
\title{Razhroščevalnik\\
    \small Seminarska naloga pri predmetu Sistemska programska oprema
    \\
    \small Mentor: doc. Tomaž Dobravec}
\date{\today}

\begin{document}

\maketitle
\thispagestyle{empty}

\begin{abstract}
	Predstavitev implementacije in delovanja rezhroščevalnika na različnih platformah in za različne jezike.
\end{abstract}

\section{Uvod}

Razhroščevalnik je program, ki nam omogoča testiranje in upravljanje nekega drugega programa. Omogoča nam branje in pisanje spomina in registrov, poljubno ustavljanje programa, izvajanje po vrsticah ali ukazih in drugo.
Poznamo različne razhroščevalnike, kot so GDB, LLDB, x64dbg, \ldots

\section{Razhroščevalnik na sistemu Linux}

Na Unix in Unix-like sistemih je priskrbljen sistemski klic
\begin{minted}{c}
    long ptrace(enum __ptrace_request op, pid_t pid, void *addr, void *data);    
\end{minted}
, ki nam omogoča, da lahko dostopamo do prcesa podanega s \code{pid}. Da pa bo sistemski klic uspel, mora proces, do katerega želimo dostopati, dovoliti dostop do njega.
To pa lahko naredi s klicem sistemske funkcije \code{ptrace}:
\begin{minted}{c}
    // pid, addr in data argumenti so ignorirani
    ptrace(PTRACE_TRACEME, 0, NULL, NULL);
\end{minted}

Po tem bo razhroščevalnik lahko dostopal do tega procesa.

Razhroščevalnik lahko sam ustvari proces, ki potem pokliče \code{ptrace} z argumentom \code{PTRACE_TRACEME},

\begin{minted}{c}
    int pid = fork();
    if (pid == 0) {
        ptrace(PTRACE_TRACEME, 0, NULL, NULL);
        execve(...);
    }
\end{minted}

lahko pa se priklopi na nek obstoječi proces z uporabo \code{PTRACE_ATTACH}, ki pošlje signal \code{SIGSTOP}, da se proces ustavi, ali pa \code{PTRACE_SEIZE}, ki ne ustavi procesa.

\begin{minted}{c}
    ptrace(PTRACE_ATTACH, pid, NULL, NULL);
    // ali
    ptrace(PTRACE_SEIZE, pid, NULL, PTRACE_O_flags);
\end{minted}


\section{Razhroščevalnik na sistemu Windows}


\section{Razhroščevalnik za jezik Java / Python ?}


\end{document}